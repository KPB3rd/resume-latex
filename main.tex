%%%%%%%%%%%%%%%%%
% This is an sample CV template created using altacv.cls
% (v1.6.4, 13 Nov 2021) written by LianTze Lim (liantze@gmail.com). Now compiles with pdfLaTeX, XeLaTeX and LuaLaTeX.
%
%% It may be distributed and/or modified under the
%% conditions of the LaTeX Project Public License, either version 1.3
%% of this license or (at your option) any later version.
%% The latest version of this license is in
%%    http://www.latex-project.org/lppl.txt
%% and version 1.3 or later is part of all distributions of LaTeX
%% version 2003/12/01 or later.
%%%%%%%%%%%%%%%%

%% Use the "normalphoto" option if you want a normal photo instead of cropped to a circle
% \documentclass[10pt,a4paper,normalphoto]{altacv}

\documentclass[10pt,a4paper,ragged2e,withhyper]{altacv}
%% AltaCV uses the fontawesome5 and packages.
%% See http://texdoc.net/pkg/fontawesome5 for full list of symbols.

% Change the page layout if you need to
\geometry{left=1.25cm,right=1.25cm,top=1.5cm,bottom=1.5cm,columnsep=1.2cm}

% The paracol package lets you typeset columns of text in parallel
\usepackage{paracol}

% Change the font if you want to, depending on whether
% you're using pdflatex or xelatex/lualatex
\ifxetexorluatex
  % If using xelatex or lualatex:
  %\setmainfont{Roboto Slab}
  %\setsansfont{"Lato"}
  \renewcommand{\familydefault}{\sfdefault}
\else
  % If using pdflatex:
  \usepackage[rm]{roboto}
  \usepackage[defaultsans]{lato}
  % \usepackage{sourcesanspro}
  \renewcommand{\familydefault}{\sfdefault}
\fi


% Change the colours if you want to
\definecolor{SlateGrey}{HTML}{2E2E2E}
\definecolor{LightGrey}{HTML}{666666}
\definecolor{DarkPastelRed}{HTML}{450808}
\definecolor{PastelRed}{HTML}{8F0D0D}
\definecolor{NiceBlue}{HTML}{528AAE}
\definecolor{DarkBlue}{HTML}{1E3F66}
\definecolor{GoldenEarth}{HTML}{E7D192}
\colorlet{name}{black}
\colorlet{tagline}{NiceBlue}
\colorlet{heading}{DarkPastelRed}
\colorlet{headingrule}{DarkBlue}
\colorlet{subheading}{NiceBlue}
\colorlet{accent}{NiceBlue}
\colorlet{emphasis}{SlateGrey}
\colorlet{body}{LightGrey}

% Change some fonts, if necessary
\renewcommand{\namefont}{\Huge\rmfamily\bfseries}
\renewcommand{\personalinfofont}{\footnotesize}
\renewcommand{\cvsectionfont}{\LARGE\rmfamily\bfseries}
\renewcommand{\cvsubsectionfont}{\large\bfseries}


% Change the bullets for itemize and rating marker
% for \cvskill if you want to
\renewcommand{\itemmarker}{{\small\textbullet}}
\renewcommand{\ratingmarker}{\faCircle}

%% Use (and optionally edit if necessary) this .tex if you
%% want to use an author-year reference style like APA(6)
%% for your publication list
\input{pubs-authoryear}

%% Use (and optionally edit if necessary) this .tex if you
%% want an originally numerical reference style like IEEE
%% for your publication list
% \input{pubs-num}

%% sample.bib contains your publications
\addbibresource{refs.bib}

\begin{document}
\name{Kenny Bowers}
% \tagline{Your Position or Tagline Here}
%% You can add multiple photos on the left or right
% \photoR{2.8cm}{Globe_High}
% \photoL{2.5cm}{Yacht_High,Suitcase_High}

\personalinfo{%
  % Not all of these are required!
  \email{kbowers3rd@gmail.com}
  \phone{864.940.6145}
%   \mailaddress{Åddrésş, Street, 00000 Cóuntry}
  \location{Atlanta, GA}
%   \homepage{linkedin.com/in/kenny-bowers/}
%   \twitter{@twitterhandle}
  \linkedin{www.linkedin.com/in/kenny-bowers/}
%   \github{your_id}
%   \orcid{0000-0000-0000-0000}
  %% You can add your own arbitrary detail with
  %% \printinfo{symbol}{detail}[optional hyperlink prefix]
  % \printinfo{\faPaw}{Hey ho!}[https://example.com/]
  %% Or you can declare your own field with
  %% \NewInfoFiled{fieldname}{symbol}[optional hyperlink prefix] and use it:
  % \NewInfoField{gitlab}{\faGitlab}[https://gitlab.com/]
  % \gitlab{your_id}
  %%
  %% For services and platforms like Mastodon where there isn't a
  %% straightforward relation between the user ID/nickname and the hyperlink,
  %% you can use \printinfo directly e.g.
  % \printinfo{\faMastodon}{@username@instace}[https://instance.url/@username]
  %% But if you absolutely want to create new dedicated info fields for
  %% such platforms, then use \NewInfoField* with a star:
  % \NewInfoField*{mastodon}{\faMastodon}
  %% then you can use \mastodon, with TWO arguments where the 2nd argument is
  %% the full hyperlink.
  % \mastodon{@username@instance}{https://instance.url/@username}
}

\makecvheader
%% Depending on your tastes, you may want to make fonts of itemize environments slightly smaller
% \AtBeginEnvironment{itemize}{\small}

%% Set the left/right column width ratio to 6:4.
\columnratio{0.6}

% Start a 2-column paracol. Both the left and right columns will automatically
% break across pages if things get too long.
\begin{paracol}{2}
\cvsection{Experience}

\cvevent{Senior Autonomy Engineer - Motion Planning}{Anduril Industries}{Oct 2020 -- Current}{Atlanta, GA}
\begin{itemize}
\item Began employment at Area-I (Anduril's first major acquisition a few months later) owning the motion planning capabilities of Altius, a fixed-wing tube-launched drone. This included researching, implementing, and benchmarking novel and state-of-the-art algorithms being delivered to multiple critical programs.
\item Initiated the transition to being a company-wide Motion Planning SME, driving adoption of Altius' algorithms to other Anduril products which improved delivery time, reliability, and minimized risk for multiple programs.
\item Engineering Lead on multiple programs, breaking down large problems into tasks for the various subteams involved. My goal is to first define MVP system interfaces that enable iteration and contained maturation for each component as you receive feedback from testing and demonstration.
\end{itemize}

\divider

\cvevent{Autonomy Research Engineer}{Georgia Tech Research Institute}{July 2017 -- Oct 2020}{Atlanta, GA}
\begin{itemize}
\item Scoped the design and C++ implementation of real-time motion planning algorithms for multi-agent aerial systems, delivering to various DoD customers on high-profile programs. This involved decomposing problems and assigning tasks to the team.
\item Repeatedly demonstrated taking vague or near-zero requirements from customers, and proposing a scoped problem to investigate. One project resulted in three publications on fundamental research in bio-inspired swarm algorithms in a one year span, and another increased budget by an order of magnitude due to strong interest by the customer.
\end{itemize}

\divider

\cvevent{Software Engineer}{Boeing Research and Technology}{June 2014 -- July 2017}{Charleston, SC}
\begin{itemize}
\item Developed algorithms and simulations for robot arm path planning and inkjet control to enable painting artwork directly onto contoured aircraft surfaces. The project was showcased during a POTUS visit, and resulted in four patents on robotic control and airplane surface inspection as it neared production-trial readiness.
\item Implemented surface inspection tooling of airplane surfaces to geometrically associate robot arm position and lidar sensor readings to compare against the expected results. This flagged micrometer anomalies and would warn the operator, detecting defects that would have been missed.
\item Volunteered to manage the collaboration between Boeing and Clemson University to sponsor Clemson’s ECE Senior Project. This included designing the project challenge for the students and judging the results at the demonstration.
\end{itemize}

% \cvsection{Projects}

% \cvevent{Project 1}{Funding agency/institution}{}{}
% \begin{itemize}
% \item Details
% \end{itemize}

% \divider

% \cvevent{Project 2}{Funding agency/institution}{Project duration}{}
% A short abstract would also work.

\medskip

% \cvsection{A Day of My Life}

% % Adapted from @Jake's answer from http://tex.stackexchange.com/a/82729/226
% % \wheelchart{outer radius}{inner radius}{
% % comma-separated list of value/text width/color/detail}
% \wheelchart{1.5cm}{0.5cm}{%
%   6/8em/accent!30/{Sleep,\\beautiful sleep},
%   3/8em/accent!40/Hopeful novelist by night,
%   8/8em/accent!60/Daytime job,
%   2/10em/accent/Sports and relaxation,
%   5/6em/accent!20/Spending time with family
% }

% use ONLY \newpage if you want to force a page break for
% ONLY the current column
% \newpage


%% Switch to the right column. This will now automatically move to the second
%% page if the content is too long.
\switchcolumn

% \cvsection{Most Proud of}

% \cvachievement{\faTrophy}{Fantastic Achievement}{and some details about it}

% \divider

% \cvachievement{\faHeartbeat}{Another achievement}{more details about it of course}

% \divider

% \cvachievement{\faHeartbeat}{Another achievement}{more details about it of course}

\cvsection{Strengths}

\cvtag{C++}
\cvtag{Python}
\cvtag{Docker}
\cvtag{CMake}

\divider\smallskip

\cvtag{Algorithm Design}
\cvtag{System Design}\\
\cvtag{Motion Planning}
\cvtag{Computational Geometry}\\
\cvtag{Machine Learning}
\cvtag{Computer Vision}

% \cvsection{Languages}

% \cvskill{English}{5}
% % \divider
% \cvskill{Spanish}{4}
% % \divider
% \cvskill{German}{3.5} %% Supports X.5 values.

%% Yeah I didn't spend too much time making all the
%% spacing consistent... sorry. Use \smallskip, \medskip,
%% \bigskip, \vspace etc to make adjustments.
\medskip

\cvsection{Education}

\cvevent{M.Sc.\ in Computer Science}{Georgia Institute of Technology}
{2016 - 2018 (part-time while working full-time)}{}
{Robotics and Computational Perception}{}

\divider

\cvevent{B.Sc.\ in Computer Engineering}{Clemson University}{2010 -- 2014}{}


\cvsection{Publications}

\nocite{*}

% \printbibliography[heading=pubtype,title={\printinfo{\faBook}{Books}},type=book]

% \divider

\printbibliography[heading=pubtype,title={\printinfo{\faFile*[regular]}{Patents}},type=article]

\divider

\printbibliography[heading=pubtype,title={\printinfo{\faUsers}{Conference Publications}},type=inproceedings]


\end{paracol}


\end{document}
